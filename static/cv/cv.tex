% Основной документ резюме
\documentclass[a4paper,12pt]{article}
% Пакеты для оформления страницы и отступов
\usepackage[a4paper,margin=0.8in]{geometry}
\usepackage{parskip}              % Отступы между абзацами
\usepackage{titlesec}        % Настройка заголовков секций
\usepackage{enumitem}        % Гибкая настройка списков
\usepackage[hidelinks]{hyperref}  % Гиперссылки без рамки
\usepackage{xcolor}          % Цветовые определения
\usepackage{fontspec}        % Работа со шрифтами в XeLaTeX
\usepackage{polyglossia}     % Многоязычная поддержка
\usepackage{ragged2e}        % Выравнивание текста
\usepackage{environ}         % Создание собственных окружений
\usepackage{emoji}

% Языковые настройки
\setmainlanguage{russian}

% Шрифты
\setmainfont{Libertinus Serif}      % Основной текст
\newfontfamily\currency{Noto Serif} % Шрифт для валюты и специальных символов

% Макрос для символа рубля
\newcommand{\ruble}{{\currency\symbol{"20BD}}}

% Окружение для блока опыта работы (с оригинальными vspace)
\NewEnviron{ExperienceBlock}[3]{%
  % #1 — Должность, #2 — Компания, #3 — Период
  \textbf{#1 | #2}

  \vspace{0.075em}

  \begin{raggedright}
    {\color{darkgray}\fontsize{10pt}{10pt}\selectfont #3}
  \end{raggedright}
  \vspace{0.45em}
  \RaggedRight
  \BODY
  \vspace{0.8em}
}

% Макрос для записи образования
\newcommand{\educationEntry}[4]{%
  \begin{raggedright}
    \textbf{#1}\\
    \textbf{#2}\\
    #3\\
    % #4\\
  \end{raggedright}
  \vspace*{0.75em}
}

% Макрос для записи проектов
\newcommand{\ProjectEntry}[3]{%
  % #1 — Название, #2 — Технологии, #3 — Описание
  \textbf{#1} -- \emph{#2}\\
  #3\\
}

% Макрос для групп навыков
\newcommand{\SkillGroup}[2]{%
  % #1 — Группа, #2 — Список навыков
  \RaggedRight
  \textbf{#1:}\\ #2\\
  \vspace*{0.4em}
}

% Окружение для отключения переносов
\newenvironment{nohyphens}{%
  \begingroup\hyphenpenalty=10000\exhyphenpenalty=10000\sloppy
}{%
  \endgroup
}

% Формат заголовков
\titleformat{\section}{\large\bfseries}{}{0em}{}[\titlerule]
\titleformat{\subsection}[runin]{\bfseries}{}{0em}{}[---]

% Расстояние между колонками
\setlength{\columnsep}{1cm}

% Отключаем колонтитулы
\pagestyle{empty}

% Макрос для правой колонки
\newcommand{\rc}[0]{
\begin{minipage}[t]{0.33\textwidth}
  \begin{flushright}
    {\LARGE \textbf{Даниил Шило}}\\
    \vspace*{0.5em}
    \href{mailto:tokiory.work@gmail.com}{tokiory.work@gmail.com}\\
    \href{https://github.com/tokiory}{github.com/tokiory}\\
    \href{https://t.me/tokiory}{t.me/tokiory}\\
    \vspace*{1em}
  \end{flushright}

  \section*{Скиллы}
  \begin{nohyphens}
    \SkillGroup{Языки}{\mbox{Javascript, Typescript, Go}}
    \SkillGroup{Фронтенд}{React, Vue, Nuxt, Next.js, Svelte, SvelteKit}
    \SkillGroup{Бэкенд}{Bun, Express.js, Gin}
    \SkillGroup{Базы данных}{PostgreSQL, Supabase, Firebase}
    \SkillGroup{DevOps}{Docker, Github Actions, Swarm, GitlabCI, Devpod}
    \SkillGroup{Cloud}{AWS}
  \end{nohyphens}
  *Все библиотеки и фремворки с которыми я работал можно увидеть в конце документа


  \section*{Образование}
  \educationEntry{Магистр}{Программная инженерия}{Луганский Государственный Университет им. В. Даля}{2023 -- 2025}
  \educationEntry{Бакалавр}{Программная инженерия}{Луганский Государственный Университет им. В. Даля}{2019 -- 2023}

\end{minipage}
}

\begin{document}

\vspace*{-2em}

\begin{minipage}[t]{0.62\textwidth}
  %%%% ЛЕВАЯ КОЛОНКА
  \vspace*{-0.22em}
  \section*{О себе}
 Фулл-стек разработчик с 5-летним опытом в создании высоконагруженных веб-платформ и приложений.

  \vspace{0.8em}

 Специализируюсь на архитектуре, рефакторинге, менторстве и разработке корпоративных решений на базе современных JavaScript/TypeScript-стеков.

  \vspace{0.8em}
  Участвовал в разработке и оптимизации мессенджеров, платформ для онлайн-совещаний, e-commerce решений, обучающих систем, таск-трекеров и внутренних инструментов.

  \vspace*{0.8em}

  \section*{Опыт работы}

  \begin{ExperienceBlock}{Фуллстек-разработчик}{HoffTech}{2023-2025 / Full Time / Core Team}
    \begin{itemize}[leftmargin=*]
      \item Занимался переходом основного проекта с Vue 3 на Nuxt 3;
      \item Создал парсер для переменных из Figma\\ и интегрировал его в UI Kit;
      \item Составил план для перехода и совместно с командой реализовал плавный переход на Typescript в огромной кодовой базе (\textgreater1000 компонентов);
      \item Учавствовал в разработке системы кэширования и интеграции Nuxt 3 с Redis;
      \item Помогал с написанием сторей и интеграцией Storybook в UI Kit, а также написанием тестов на Vitest;
      \item Разработал стандарт по стайлингу Typescript-кода;
      \item Оптимизировал платформозависимый код (кроссбраузерность/кроссплатформенность);
      \item Оптимизировал показатели Core Web Vitals;
      \item Учавствовал в митах об архитектурных решениях,\\а также лобировал решения и практики;
      \item Переписывал большие легаси-модули, а также выполнял переход от Options API к Composition API;
      \item Разработал кастомный механизм фетчинга данных (мультиклиент) на основе ofetch;
      \item Внедрял логирование ошибок с помощью Sentry в SSR-версию приложения, а также написал \\кастомные плагины для обработки ошибок в Nuxt и Nitro;
      \item Мигрировал сторы с Vuex на Pinia;
          \item Фиксил множество багов и занимался ревью команд фронтендеров;
    \end{itemize}
  \end{ExperienceBlock}
\end{minipage}
\hfill
%%%% ПРАВАЯ КОЛОНКА
\rc

\begin{minipage}[t]{1\textwidth}
    \begin{itemize}[leftmargin=*]
      \item Реализовал документацию на базе Redocly для Swagger-файлов;
      \item Консультировал других разработчиков в вопросах относительно SSR, Nuxt и Typescript;
        \end{itemize}

\vspace*{0.8em}

\begin{ExperienceBlock}{Фронтенд-разработчик}{MTS}{2022-2023 / Full Time / Web-Mobile Team}
    \begin{itemize}[leftmargin=*]
      \item Разрабатывал мультиплатформенное приложение на базе WebView, Ionic и Capacitor;
     \item Учавствовал в Performance Debugging на различных платформах (iOS/Android);
    \item Переписывал легаси Javascript-код на Typescript;
    \item Рефакторил множество модулей;
    \item Предлагал и согласовывал архитектурные решения с командой;
    \item Дебажил баги в среде XCode (эмулятор);
    \item Консультировал других разработчиков;
    \item Проводил код-ревью;
    \end{itemize}
\end{ExperienceBlock}




\begin{ExperienceBlock}{Фронтенд-разработчик}{Гранд Сервис Экспресс}{06.2021 -- 12.2021 / Full Time / Solo}
  \begin{itemize}[leftmargin=*]
    \item Разработал PWA-приложение для проводников;
    \item Разработал систему отказоустойчивости при потере интернет-соединения (пул ивентов);
    \item Написал документацию к кодовой базе;
    \item Работал с API на базе GraphQL;
    \item Написал E2E-тесты на возможные пользовательские сценарии (Playwright);
    \item Реализовал мост между веб-приложением (PWA) и хостом (Android-приложением);
    \item Активно взаимодействовал с техническим писателем, системным архитектором \\ и аналитиком;
  \end{itemize}
\end{ExperienceBlock}

\begin{ExperienceBlock}{Фронтенд-разработчик}{Черная жемчужина}{01.2021 -- 05.2021 / Full Time / Core Team}
  \begin{itemize}[leftmargin=*]
    \item Занимался разработкой лендинга для компании с нуля;
    \item Работал с Three.js для создания интерактивного элемента в блоке-герое;
    \item Тимлидил команду, помогал PM в декомпозиции работы, а также распределял задачи;
    \item Менторил разработчиков;
    \item Настраивал ESLint, Prettier и другие утилиты для улучшения DX;
  \end{itemize}
\end{ExperienceBlock}

\begin{ExperienceBlock}{Фронтенд-разработчик}{Blogman}{2020-2021 / Full Time / Frontend Team}
  \begin{itemize}[leftmargin=*]
    \item Разработал фичу "Спуститься к последнему сообщению";
    \item Написал плагин для Electron, который выполнял автообновление на UNIX-подобных системах;
    \item Писал документацию к кодовой базе;
    \item Активно учавствовал в ретроспективах;
    \item Написал часть функционала для меню списка чатов;
    \item Сверстал новую версию карточки пользователя;
    \item Менторил других разработчиков;
    \item Проводил код-ревью;
  \end{itemize}
\end{ExperienceBlock}
\end{minipage}
\begin{minipage}[t]{1\textwidth}
\section*{Пет-проекты}

 \ProjectEntry{Developer Log}{Nuxt 3, Typescript}{\href{https://developer-log.vercel.app/}{Личный блог} с открытым исходным кодом (\href{https://github.com/tokiory/developer-log}{GitHub}).}

  \ProjectEntry{Intquest}{React, Typescript, TailwindCSS, Nanostores}{Сервис подбора вопросов на собеседованиях (\href{https://github.com/tokiory/intquest}{GitHub}).}

  \ProjectEntry{Termfolio}{Vue 3, Vite, Typescript}{Эмулятор терминала в браузере: \href{https://tokiory.vercel.app/}{демо} и \href{https://github.com/tokiory/termfolio}{код}.}

  \ProjectEntry{Neovim Boilerplate}{Lua}{Базовый конфиг Neovim с \textasciitilde100+ звёздами на \href{https://github.com/tokiory/neovim-boilerplate}{GitHub}.}

\end{minipage}

\section*{Дополнительные технологии}
\begin{nohyphens}
  \SkillGroup{Сборщики}{Vite, Webpack, Parcel, ESBuild, Rollup, Gulp}

  \SkillGroup{API}{REST API, GraphQL, WebSocket, WebRTC, gRPC}

  \SkillGroup{Стейт-менеджеры}{Redux + RTK, Pinia, Vuex, Nanostores, Zustand}

  \SkillGroup{Кроссплатформенные фреймворки}{Electron, Tauri, Ionic, Capacitor}

  \SkillGroup{Тестирование}{Jest, Vitest, Playwright, Mocha, Chai, Supertest}

  \SkillGroup{Мониторинг}{Sentry, Grafana}

  \SkillGroup{Безопасность}{JWT, SSL/TLS}

  \SkillGroup{UI Библиотеки и CSS-фреймворки}{Material UI, Chakra UI, shadcn/ui, PrimeVue, Tailwind, Bootstrap, Vuetify, Quasar}

  \SkillGroup{Документация}{Swagger, Storybook, JSDoc, TypeDoc, Docz, VuePress, Docusaurus}

  \SkillGroup{Линтеры/Форматтеры}{ESLint, Prettier, SonarQube, Stylelint}

  \SkillGroup{Контейнеризация}{Docker, Docker Swarm, Kubernetes, OrbStack, Podman}

  \SkillGroup{Метрики и аналитика}{Google Analytics, Yandex Metrica, Umami}
\end{nohyphens}

\end{document}
